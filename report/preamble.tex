\documentclass[conference]{IEEEtran}

\usepackage{cite}
\ifCLASSINFOpdf
\usepackage[pdftex]{graphicx}
\graphicspath{{../script/fig/}{./figures/}}
\DeclareGraphicsExtensions{.pdf,.jpeg,.png, .jpg, .PNG}
\else
\fi
\usepackage{amsmath,bm}
\interdisplaylinepenalty=2500
\ifCLASSOPTIONcompsoc
\usepackage[caption=false,font=normalsize,labelfont=sf,textfont=sf]{subfig}
\else
\usepackage[caption=false,font=footnotesize]{subfig}
\fi
\usepackage{url}
\usepackage{amsfonts} % for \mathbb{}
\usepackage{listings} % For writing math in verbatim
\usepackage{enumerate}

\usepackage[showlabels,sections,floats,textmath,displaymath]{preview}
\usepackage{amssymb} %% For showing special symbols like \bigstar
\usepackage{pifont}  %% For showing circled numbers
\usepackage{array}
%\usepackage{algorithm}
\usepackage{algpseudocode}
\usepackage{ifthen}
\usepackage{dirtree}
%\setlength{\DTbaselineskip}{13pt}
%\renewcommand*\DTstyle{}
%\DTsetlength{.2em}{1em}{.2em}{.4pt}{2pt}

\newcommand{\mathnew}{\mathit} 
% for nice reference 
\newcommand{\fsp}{{\;}}
\newcommand{\Fig}{\figurename \fsp}
\newcommand{\Sec}{Sec.\fsp}
\newcommand{\Eq}{Eq.\fsp}
\newcommand{\Tab}{Tab.\fsp}
\newcommand{\Alg}{Alg.\fsp}

\DeclareMathOperator*{\argminB}{argmin} 

\hyphenation{temperature} % correct bad hyphenation here

\usepackage{tikz}
\usetikzlibrary{arrows}
\usetikzlibrary{intersections}
\usetikzlibrary{calc, quotes}
\usetikzlibrary{external}
\tikzexternalize[prefix=tikz_figures/]%
\pgfdeclarelayer{bg}    % declare background layer
\pgfsetlayers{bg,main} % set the order of the layers (main is the standard layer)
\usetikzlibrary{patterns}

\newcommand{\etal}{\textit{et al}. }
\newcommand{\ie}{\textit{i}.\textit{e}. }
\newcommand{\eg}{\textit{e}.\textit{g}. }

%%
%% Check this post for how to write properly-sized ceiling operator
%% https://tex.stackexchange.com/a/367035/172530
%%
\usepackage{mathtools}
\DeclarePairedDelimiter{\ceil}{\lceil}{\rceil}

%% For text color
\usepackage{xcolor}

%% For drawing boxes around text
%% https://tex.stackexchange.com/a/234195/172530
\newcommand\textbox[2]{%
  \tikz[baseline]\node[%
  inner ysep=0pt, 
  inner xsep=2pt, 
  anchor=text, 
  rectangle, 
  rounded corners=1mm,
  #1] {\strut#2};%
}

%% package for highlighting text
%% Usage: \hl{text to be highlighted}
\usepackage{soul}
\sethlcolor{yellow}

%% underlining for emphasis
%% Examples:
%% \uline{important} underline
%% \uuline{urgent} double-underline
%% \uwave{boat} wavy underline like
%% \sout{wrong} line struck through word
%% \xout{removed} removed
%% \dashuline{dashing} dashed
%% \dotuline{dotty} dotted
\usepackage[normalem]{ulem}

%% define a red check mark
\newcommand\redcheck{
  {\textcolor{red}{\ding{52}}}
}

\usepackage{dblfloatfix}    % To enable figures at the bottom of page

\usepackage{wrapfig}
\usepackage{siunitx}

% Overlay question mark on equal mark
\def\qeq{\mathrel{%
    \mathchoice{\QEQ}{\QEQ}{\scriptsize\QEQ}{\tiny\QEQ}%
}}
\def\QEQ{{%
    \setbox0\hbox{=}%
    \rlap{\hbox to \wd0{\hss?\hss}}\box0
  }}


%% Enabling this package will automatically align the text in the two columns vertically
%\usepackage{widetext}


%% https://tex.stackexchange.com/questions/2441/how-to-add-a-forced-line-break-inside-a-table-cell
\usepackage{makecell} %% For breaking lines in table cells

\usepackage{amsthm}
\theoremstyle{definition}
\newtheorem{definition}{Definition}

\theoremstyle{proposition}
\newtheorem{proposition}{Proposition}

\DeclareMathOperator*{\argmax}{arg\,max}
\DeclareMathOperator*{\argmin}{arg\,min}

%% It is tricky to use the algorithm2e package in the ieeetran document
%% Check this post for tweaks
%% https://tex.stackexchange.com/a/147625
\usepackage[ruled,norelsize,linesnumbered, vlined, longend]{algorithm2e}
\makeatletter
\newcommand{\removelatexerror}{\let\@latex@error\@gobble}
\makeatother
